\documentclass[11pt,a4paper]{article}

% Packages
\usepackage[utf8]{inputenc}
\usepackage[T1]{fontenc}
\usepackage{amsmath,amssymb,amsfonts}
\usepackage{graphicx}
\usepackage{booktabs}
\usepackage{multirow}
\usepackage{hyperref}
\usepackage{algorithm}
\usepackage{algorithmic}
\usepackage{listings}
\usepackage{xcolor}
\usepackage{geometry}
\usepackage{caption}
\usepackage{subcaption}
\usepackage{natbib}
\usepackage{float}

\geometry{margin=1in}

% Code listing style
\lstset{
    basicstyle=\ttfamily\small,
    breaklines=true,
    frame=single,
    numbers=left,
    numberstyle=\tiny,
    keywordstyle=\color{blue},
    commentstyle=\color{green!60!black},
    stringstyle=\color{red},
}

\title{\textbf{CIO-Agent FAB++: A Dynamic Multi-Dimensional Benchmark\\for Evaluating AI Finance Agents}}

\author{
    Team AgentBusters\\
    AgentBeats Competition 2026\\
    \texttt{https://github.com/yxc20089/AgentBusters}
}

\date{\today}

\begin{document}

\maketitle

\begin{abstract}
We present CIO-Agent FAB++ (Finance Agent Benchmark Plus Plus), a comprehensive evaluation framework for assessing AI agents on financial analysis tasks. Unlike static benchmarks, FAB++ dynamically generates evaluation tasks across 18 categories spanning fundamental analysis, quantitative reasoning, options trading, and risk management. The system implements a novel multi-dimensional scoring methodology that evaluates agents on macro thesis quality, fundamental accuracy, execution methodology, and adversarial robustness. We introduce the Options Alpha Challenge, a specialized evaluation track that tests agents on Black-Scholes pricing, Greeks analysis, and multi-leg strategy construction. Our framework leverages the Agent-to-Agent (A2A) protocol for standardized communication and Model Context Protocol (MCP) servers for real-time financial data access. Experimental results demonstrate the effectiveness of our evaluation methodology in distinguishing agent capabilities across diverse financial reasoning tasks.
\end{abstract}

\textbf{Keywords:} AI Agents, Finance Benchmark, Options Trading, Agent Evaluation, A2A Protocol, MCP

\section{Introduction}

The rapid advancement of large language models (LLMs) has enabled the development of sophisticated AI agents capable of performing complex financial analysis tasks \citep{brown2020language}. However, evaluating these agents presents significant challenges: financial reasoning requires numerical precision, temporal awareness, and domain expertise that traditional NLP benchmarks fail to capture adequately.

Existing finance benchmarks suffer from several limitations:
\begin{enumerate}
    \item \textbf{Static evaluation}: Fixed question sets become memorized by models during training, leading to inflated performance metrics.
    \item \textbf{Single-dimensional scoring}: Most benchmarks evaluate only answer correctness, ignoring reasoning quality and methodology.
    \item \textbf{Lack of temporal constraints}: Agents may inadvertently access future information, violating realistic trading scenarios.
    \item \textbf{Limited options coverage}: Few benchmarks evaluate quantitative finance skills like derivatives pricing and risk management.
\end{enumerate}

We address these limitations with CIO-Agent FAB++, a dynamic benchmark system that:
\begin{itemize}
    \item Generates novel evaluation tasks from real financial data with temporal locking
    \item Evaluates agents across multiple dimensions including macro reasoning, fundamental accuracy, and execution quality
    \item Introduces adversarial debate to test conviction and robustness
    \item Provides comprehensive options trading evaluation with Black-Scholes pricing verification
\end{itemize}

\section{Related Work}

\subsection{Financial Benchmarks}

The Finance Agent Benchmark (FAB) \citep{fab2025} introduced structured evaluation of AI agents on earnings analysis tasks. BizFinBench \citep{bizfinbench2025} expanded coverage to include Chinese financial markets and multi-turn reasoning. However, these benchmarks use static question sets vulnerable to data contamination.

\subsection{Agent Communication Protocols}

The Agent-to-Agent (A2A) protocol \citep{a2a2025} standardizes communication between AI agents, enabling interoperability across different implementations. The Model Context Protocol (MCP) \citep{mcp2024} provides a unified interface for agents to access external tools and data sources.

\subsection{Options Pricing Models}

The Black-Scholes-Merton model \citep{black1973pricing, merton1973theory} remains the foundation for options pricing. Extensions include stochastic volatility models \citep{heston1993closed} and jump-diffusion processes \citep{merton1976option}.

\section{System Architecture}

\subsection{Overview}

FAB++ implements a Green Agent (evaluator) and Purple Agent (finance analyst) architecture following the A2A protocol specification. Figure \ref{fig:architecture} illustrates the system components.

\begin{figure}[H]
\centering
\begin{verbatim}
+------------------------------------------------------------------+
|                    AgentBusters System                            |
+------------------------------------------------------------------+
|  +---------------+      A2A Protocol      +---------------+       |
|  | Green Agent   |<--------------------->| Purple Agent  |       |
|  | (Evaluator)   |                       | (Analyst)     |       |
|  | Port: 9109    |                       | Port: 9110    |       |
|  +-------+-------+                       +-------+-------+       |
|          |                                       |                |
|          |     +-----------------------------+   |                |
|          |     |       6 MCP Servers         |   |                |
|          |     | SEC EDGAR  | Yahoo Finance  |   |                |
|          |     | Sandbox    | Options Chain  |   |                |
|          |     | Trading Sim| Risk Metrics   |   |                |
|          |     +-----------------------------+   |                |
+------------------------------------------------------------------+
\end{verbatim}
\caption{FAB++ System Architecture}
\label{fig:architecture}
\end{figure}

\subsection{Green Agent (Evaluator)}

The Green Agent serves as the benchmark orchestrator, responsible for:
\begin{itemize}
    \item Dynamic task generation from financial data templates
    \item Multi-dimensional response evaluation
    \item Adversarial counter-argument generation
    \item Alpha Score computation
\end{itemize}

\subsection{Purple Agent (Finance Analyst)}

The Purple Agent represents the system under test, implementing:
\begin{itemize}
    \item Financial data retrieval via MCP servers
    \item LLM-powered analysis generation
    \item Options strategy construction
    \item Risk assessment and position sizing
\end{itemize}

\subsection{MCP Server Infrastructure}

We deploy six MCP servers providing specialized financial capabilities:

\begin{table}[H]
\centering
\caption{MCP Server Specifications}
\label{tab:mcp_servers}
\begin{tabular}{llp{6cm}}
\toprule
\textbf{Server} & \textbf{Port} & \textbf{Capabilities} \\
\midrule
SEC EDGAR & 8101 & 10-K/10-Q filings, XBRL parsing, temporal locking \\
Yahoo Finance & 8102 & Real-time quotes, historical data, lookahead detection \\
Python Sandbox & 8103 & Secure code execution for numerical computations \\
Options Chain & 8104 & Black-Scholes pricing, Greeks calculation, IV surface \\
Trading Simulator & 8105 & Paper trading, slippage modeling, P\&L tracking \\
Risk Metrics & 8106 & VaR computation, Sharpe/Sortino ratios, stress testing \\
\bottomrule
\end{tabular}
\end{table}

\section{Evaluation Methodology}

\subsection{Task Categories}

FAB++ evaluates agents across 18 categories organized into three tiers:

\subsubsection{Core Finance (6 categories)}
\begin{itemize}
    \item \textbf{Beat or Miss}: Earnings surprise detection against analyst consensus
    \item \textbf{Macro Analysis}: Economic trend interpretation and market impact
    \item \textbf{Fundamental Analysis}: Financial statement interpretation
    \item \textbf{Quantitative Reasoning}: Numerical calculations from financial data
    \item \textbf{SEC Filing Analysis}: Information extraction from regulatory documents
    \item \textbf{Trend Analysis}: Historical pattern recognition and forecasting
\end{itemize}

\subsubsection{Options Alpha (6 categories)}
\begin{itemize}
    \item \textbf{Options Pricing}: Black-Scholes valuation and fair value assessment
    \item \textbf{Greeks Analysis}: Sensitivity calculations and hedging strategies
    \item \textbf{Strategy Construction}: Multi-leg options strategies
    \item \textbf{Volatility Trading}: IV rank/percentile analysis
    \item \textbf{P\&L Attribution}: Return decomposition by Greek exposure
    \item \textbf{Risk Management}: VaR-based position sizing
\end{itemize}

\subsubsection{Advanced (6 categories)}
\begin{itemize}
    \item \textbf{Copy Trading}: Strategy replication and signal generation
    \item \textbf{Race to 10M}: Capital growth optimization under constraints
    \item \textbf{Strategy Defense}: Adversarial robustness testing
    \item \textbf{Financial Data Description}: Structured data interpretation
    \item \textbf{Multi-turn Perception}: Context maintenance across interactions
    \item \textbf{Sentiment Analysis}: Market sentiment extraction
\end{itemize}

\subsection{Dynamic Task Generation}

Unlike static benchmarks, FAB++ generates tasks dynamically using templates populated with real financial data:

\begin{algorithm}[H]
\caption{Dynamic Task Generation}
\begin{algorithmic}[1]
\REQUIRE Template $T$, Financial Lake $\mathcal{F}$, Simulation Date $d$
\STATE Select ticker $s$ from universe $\mathcal{S}$
\STATE Lock temporal context to date $d$
\STATE Retrieve fundamental data $F_s = \mathcal{F}(s, d)$
\STATE Generate ground truth $G$ from $F_s$
\STATE Instantiate task $\tau = T(s, F_s, G, d)$
\STATE Compute rubric criteria $R$ for $\tau$
\RETURN Task $(\tau, G, R)$
\end{algorithmic}
\end{algorithm}

\subsection{Multi-Dimensional Scoring}

We evaluate responses across three primary dimensions:

\subsubsection{Role Score}
The Role Score combines weighted subscores:
\begin{equation}
\text{RoleScore} = 0.30 \cdot S_{\text{macro}} + 0.40 \cdot S_{\text{fundamental}} + 0.30 \cdot S_{\text{execution}}
\end{equation}

where:
\begin{itemize}
    \item $S_{\text{macro}}$: Macro thesis quality (semantic similarity + theme coverage)
    \item $S_{\text{fundamental}}$: Numerical accuracy against ground truth
    \item $S_{\text{execution}}$: Methodology quality and tool usage
\end{itemize}

\subsubsection{Adversarial Debate}

We introduce adversarial debate to test agent conviction:

\begin{algorithm}[H]
\caption{Adversarial Debate Protocol}
\begin{algorithmic}[1]
\REQUIRE Agent response $A$, Task $\tau$
\STATE Generate counter-argument $C$ challenging $A$
\STATE Request rebuttal $R$ from agent
\STATE Evaluate conviction: maintained, weakened, or collapsed
\STATE Compute debate multiplier $m \in [0.8, 1.2]$
\RETURN Multiplier $m$
\end{algorithmic}
\end{algorithm}

\subsubsection{Alpha Score}

The final Alpha Score combines all dimensions:
\begin{equation}
\alpha = \frac{\text{RoleScore} \times \text{DebateMultiplier}}{\ln(1 + \text{Cost}) \times (1 + \text{LookaheadPenalty})}
\end{equation}

This formulation rewards accurate, robust responses while penalizing expensive computation and temporal violations.

\section{Options Alpha Challenge}

\subsection{Black-Scholes Implementation}

The Options Chain MCP server implements the Black-Scholes-Merton model with dividend yield:

\begin{equation}
d_1 = \frac{\ln(S/K) + (r - q + \sigma^2/2)T}{\sigma\sqrt{T}}
\end{equation}
\begin{equation}
d_2 = d_1 - \sigma\sqrt{T}
\end{equation}

Call and put prices:
\begin{align}
C &= Se^{-qT}N(d_1) - Ke^{-rT}N(d_2) \\
P &= Ke^{-rT}N(-d_2) - Se^{-qT}N(-d_1)
\end{align}

where $S$ is spot price, $K$ is strike, $r$ is risk-free rate, $q$ is dividend yield, $\sigma$ is volatility, and $T$ is time to expiration.

\subsection{Greeks Calculation}

We compute the standard Greeks for evaluation:

\begin{table}[H]
\centering
\caption{Options Greeks Formulas}
\label{tab:greeks}
\begin{tabular}{lll}
\toprule
\textbf{Greek} & \textbf{Call} & \textbf{Put} \\
\midrule
Delta ($\Delta$) & $e^{-qT}N(d_1)$ & $-e^{-qT}N(-d_1)$ \\
Gamma ($\Gamma$) & $\frac{e^{-qT}n(d_1)}{S\sigma\sqrt{T}}$ & Same as call \\
Theta ($\Theta$) & $-\frac{Se^{-qT}n(d_1)\sigma}{2\sqrt{T}} - rKe^{-rT}N(d_2)$ & Complex \\
Vega ($\nu$) & $Se^{-qT}\sqrt{T}n(d_1)$ & Same as call \\
Rho ($\rho$) & $KTe^{-rT}N(d_2)$ & $-KTe^{-rT}N(-d_2)$ \\
\bottomrule
\end{tabular}
\end{table}

\subsection{Options Evaluation Scoring}

The Options Evaluator uses a four-dimensional scoring rubric:

\begin{equation}
S_{\text{options}} = 0.25 \cdot S_{\text{P\&L}} + 0.25 \cdot S_{\text{Greeks}} + 0.25 \cdot S_{\text{Strategy}} + 0.25 \cdot S_{\text{Risk}}
\end{equation}

\begin{table}[H]
\centering
\caption{Options Scoring Dimensions}
\label{tab:options_scoring}
\begin{tabular}{lp{8cm}}
\toprule
\textbf{Dimension} & \textbf{Evaluation Criteria} \\
\midrule
P\&L Accuracy & Max profit/loss calculations, breakeven points, probability of profit \\
Greeks Accuracy & Delta, gamma, theta, vega values within 5\% tolerance \\
Strategy Quality & Correct leg identification, strike selection rationale, structure validity \\
Risk Management & Position sizing, hedging strategy, exit criteria definition \\
\bottomrule
\end{tabular}
\end{table}

\section{Experiments}

\subsection{Experimental Setup}

We evaluated a baseline Purple Agent using GPT-4o as the underlying LLM. The agent was tested on:
\begin{itemize}
    \item Synthetic questions generated from the Financial Lake
    \item Public CSV finance Q\&A dataset
    \item Options trading tasks (iron condor, volatility, Greeks, risk)
\end{itemize}

\subsection{Results}

\begin{table}[H]
\centering
\caption{Baseline Agent Performance}
\label{tab:results}
\begin{tabular}{lrrr}
\toprule
\textbf{Dataset} & \textbf{Tasks} & \textbf{Score} & \textbf{Metric} \\
\midrule
Synthetic Questions & 2 & 60,418 & Alpha Score \\
Public CSV & 3 & 66.67\% & Accuracy \\
Options (Iron Condor) & 1 & 50.8/100 & Options Score \\
\bottomrule
\end{tabular}
\end{table}

\subsubsection{Options Evaluation Breakdown}

\begin{table}[H]
\centering
\caption{Options Task Performance (Iron Condor on SPY)}
\label{tab:options_results}
\begin{tabular}{lr}
\toprule
\textbf{Dimension} & \textbf{Score} \\
\midrule
P\&L Accuracy & 68.0/100 \\
Greeks Accuracy & 0.0/100 \\
Strategy Quality & 50.0/100 \\
Risk Management & 85.0/100 \\
\midrule
\textbf{Final Options Score} & \textbf{50.8/100} \\
\bottomrule
\end{tabular}
\end{table}

The results reveal that while the baseline agent performs well on risk management concepts (85/100), it struggles with explicit Greeks calculations (0/100), indicating a gap between conceptual understanding and numerical precision.

\subsection{Public CSV Detailed Results}

\begin{table}[H]
\centering
\caption{Public CSV Dataset Performance}
\label{tab:csv_results}
\begin{tabular}{lrr}
\toprule
\textbf{Question Category} & \textbf{Correct} & \textbf{Score} \\
\midrule
Market Analysis (US Steel Merger) & 4/4 & 100\% \\
Trends (Netflix ARPU) & 9/9 & 100\% \\
Beat or Miss (TJX Margin) & 0/2 & 0\% \\
\midrule
\textbf{Average} & 13/15 & \textbf{66.67\%} \\
\bottomrule
\end{tabular}
\end{table}

\section{Discussion}

\subsection{Key Findings}

\begin{enumerate}
    \item \textbf{Conceptual vs. Computational}: Agents demonstrate strong conceptual understanding but struggle with precise numerical calculations, particularly in options pricing.

    \item \textbf{Category Variance}: Performance varies significantly across categories, with fundamental analysis outperforming quantitative tasks.

    \item \textbf{Temporal Awareness}: Agents occasionally exhibit lookahead bias, accessing information beyond the simulation date.

    \item \textbf{Risk Management Strength}: Agents excel at qualitative risk discussions but underperform on quantitative risk metrics.
\end{enumerate}

\subsection{Limitations}

\begin{itemize}
    \item Ground truth for subjective tasks (macro analysis) relies on reference summaries
    \item Options pricing assumes Black-Scholes model validity
    \item Adversarial debate quality depends on counter-argument generation
\end{itemize}

\subsection{Future Work}

\begin{itemize}
    \item Extend to multi-agent trading simulations
    \item Incorporate stochastic volatility models
    \item Add real-time market data integration
    \item Develop specialized evaluators for emerging asset classes
\end{itemize}

\section{Conclusion}

We presented CIO-Agent FAB++, a comprehensive benchmark for evaluating AI finance agents across 18 categories spanning fundamental analysis, options trading, and risk management. Our multi-dimensional scoring methodology, combined with adversarial debate testing, provides nuanced assessment of agent capabilities beyond simple accuracy metrics. The Options Alpha Challenge introduces rigorous quantitative evaluation using Black-Scholes pricing and Greeks verification. Experimental results demonstrate the benchmark's effectiveness in revealing agent strengths and weaknesses across diverse financial reasoning tasks.

The system is publicly available at \url{https://github.com/yxc20089/AgentBusters} with Docker images for immediate deployment.

\section*{Acknowledgments}

We thank the AgentBeats Competition organizers at Berkeley RDI for inspiring this work. We acknowledge the contributions of the A2A Protocol and MCP communities for enabling standardized agent communication.

\bibliographystyle{plainnat}
\begin{thebibliography}{99}

\bibitem[Black and Scholes(1973)]{black1973pricing}
Black, F. and Scholes, M. (1973).
\newblock The pricing of options and corporate liabilities.
\newblock \emph{Journal of Political Economy}, 81(3):637--654.

\bibitem[Brown et al.(2020)]{brown2020language}
Brown, T.~B., Mann, B., Ryder, N., et al. (2020).
\newblock Language models are few-shot learners.
\newblock \emph{Advances in Neural Information Processing Systems}, 33:1877--1901.

\bibitem[Merton(1973)]{merton1973theory}
Merton, R.~C. (1973).
\newblock Theory of rational option pricing.
\newblock \emph{The Bell Journal of Economics and Management Science}, 4(1):141--183.

\bibitem[Merton(1976)]{merton1976option}
Merton, R.~C. (1976).
\newblock Option pricing when underlying stock returns are discontinuous.
\newblock \emph{Journal of Financial Economics}, 3(1-2):125--144.

\bibitem[Heston(1993)]{heston1993closed}
Heston, S.~L. (1993).
\newblock A closed-form solution for options with stochastic volatility with applications to bond and currency options.
\newblock \emph{The Review of Financial Studies}, 6(2):327--343.

\bibitem[Bigeard et al.(2025)]{fab2025}
Bigeard, A., Nashold, L., Krishnan, R., and Wu, S. (2025).
\newblock Finance Agent Benchmark: Benchmarking LLMs on Real-world Financial Research Tasks.
\newblock \emph{arXiv preprint arXiv:2508.00828}.
\newblock \url{https://arxiv.org/abs/2508.00828}.

\bibitem[Lu et al.(2025)]{bizfinbench2025}
Lu, G., Guo, X., Zhang, R., Zhu, W., and Liu, J. (2025).
\newblock BizFinBench.v2: A Unified Dual-Mode Bilingual Benchmark for Expert-Level Financial Capability Alignment.
\newblock \emph{arXiv preprint arXiv:2601.06401}.
\newblock \url{https://arxiv.org/abs/2601.06401}.

\bibitem[A2A Protocol(2025)]{a2a2025}
Google and Linux Foundation (2025).
\newblock Agent-to-Agent Protocol: An open protocol enabling communication and interoperability between opaque agentic applications.
\newblock \url{https://github.com/a2aproject/A2A}.

\bibitem[MCP(2024)]{mcp2024}
Anthropic (2024).
\newblock Model Context Protocol.
\newblock \url{https://modelcontextprotocol.io/}.

\end{thebibliography}

\appendix

\section{Alpha Score Derivation}

The Alpha Score is designed to reward accurate, robust, and efficient agent responses:

\begin{equation}
\alpha = \frac{R \cdot D}{C \cdot P}
\end{equation}

where:
\begin{itemize}
    \item $R$ = RoleScore $\in [0, 100]$
    \item $D$ = DebateMultiplier $\in [0.8, 1.2]$
    \item $C$ = $\ln(1 + \text{Cost})$ (logarithmic cost penalty)
    \item $P$ = $1 + \text{LookaheadPenalty}$ (temporal violation penalty)
\end{itemize}

The logarithmic cost penalty ensures diminishing returns for expensive computations, while the lookahead penalty harshly penalizes agents that access future information.

\section{MCP Server API Reference}

\subsection{Options Chain Server}

\begin{lstlisting}[language=Python, caption=Options Chain MCP Tools]
# Get options chain for a ticker
get_options_chain(ticker: str, expiration: str) -> dict

# Calculate Black-Scholes price
calculate_option_price(
    spot: float, strike: float, rate: float,
    volatility: float, time_to_expiry: float,
    option_type: str, dividend_yield: float
) -> dict  # Returns price and all Greeks

# Get implied volatility surface
get_iv_surface(ticker: str) -> dict

# Analyze multi-leg strategy
analyze_strategy(legs: list[dict]) -> dict
\end{lstlisting}

\subsection{Risk Metrics Server}

\begin{lstlisting}[language=Python, caption=Risk Metrics MCP Tools]
# Calculate portfolio Greeks
calculate_portfolio_greeks(positions: list[dict]) -> dict

# Calculate Value at Risk
calculate_var(
    returns: list[float], confidence: float,
    method: str  # "historical", "parametric", "monte_carlo"
) -> dict

# Run stress test
run_stress_test(
    portfolio: dict,
    scenarios: list[dict]  # e.g., {"name": "crash", "spot_change": -0.20}
) -> dict
\end{lstlisting}

\section{Evaluation Configuration}

\begin{lstlisting}[language=Python, caption=Sample Evaluation Config (YAML)]
name: "FAB++ Full Evaluation"
datasets:
  - type: synthetic
    path: data/synthetic_questions/questions.json
    limit: 50
  - type: bizfinbench
    path: data/BizFinBench.v2
    task_types: [event_logic_reasoning, financial_quantitative_computation]
    languages: [en]
    limit_per_task: 20
  - type: public_csv
    path: finance-agent/data/public.csv
    limit: 100
sampling:
  strategy: stratified
  total_limit: 100
  seed: 42
\end{lstlisting}

\end{document}
